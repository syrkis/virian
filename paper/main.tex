% This must be in the first 5 lines to tell arXiv to use pdfLaTeX, which is strongly recommended.
\pdfoutput=1
% In particular, the hyperref package requires pdfLaTeX in order to break URLs across lines.

\documentclass[11pt]{article}

% Remove the "review" option to generate the final version.
\usepackage{acl}

% Standard package includes
\usepackage{times}
\usepackage{float}
\usepackage{latexsym}

% For proper rendering and hyphenation of words containing Latin characters (including in bib files)
\usepackage[T1]{fontenc}
% For Vietnamese characters
% \usepackage[T5]{fontenc}
% See https://www.latex-project.org/help/documentation/encguide.pdf for other character sets

% This assumes your files are encoded as UTF8
\usepackage[utf8]{inputenc}

% This is not strictly necessary, and may be commented out,
% but it will improve the layout of the manuscript,
% and will typically save some space.
\usepackage{microtype}

% If the title and author information does not fit in the area allocated, uncomment the following
%
%\setlength\titlebox{<dim>}
%
% and set <dim> to something 5cm or larger.

\title{Mapping Population Reading Activity to Human Value Survey Responses}
\title{Inferring Cultural States from Population Reading Activity}

% Author information can be set in various styles:
% For several authors from the same institution:
% \author{Author 1 \and ... \and Author n \\
%         Address line \\ ... \\ Address line}
% if the names do not fit well on one line use
%         Author 1 \\ {\bf Author 2} \\ ... \\ {\bf Author n} \\
% For authors from different institutions:
% \author{Author 1 \\ Address line \\  ... \\ Address line
%         \And  ... \And
%         Author n \\ Address line \\ ... \\ Address line}
% To start a seperate ``row'' of authors use \AND, as in
% \author{Author 1 \\ Address line \\  ... \\ Address line
%         \AND
%         Author 2 \\ Address line \\ ... \\ Address line \And
%         Author 3 \\ Address line \\ ... \\ Address line}

\author{Noah Syrkis \\
  IT University of Copenhagen \\
  \texttt{noah@syrkis.com} \\}

\begin{document}
\maketitle
\begin{abstract}
    International social surveys are conducted by various institutions\footnote{The European Social Survey and the World Value Survey releases rounds every two and seven years respectively.} with yearly time granularity.
    Here a method for inferring nation-specifc distributions of answers to the European Social Survey from view-count weigthed Wikipedia reading activity is proposed.
    Specifically a neural model is train to predict the mean and variance of the five most doninant factors to to the subset of questions seen in Appendeix A for the countries seen in Appendix B.
    The method is designed to be time scale agnostic, though the experiemental setup is only able to evaluate biu-yearly accuracy (the \texttt{ESS} is conduced every second year).
    The mean and variance of the distribution of the most relevant factors of the responese matrix these can \emph{1)} be predcited for unknown countries, and \emph{2)} assign means and variance to these factors at arbitaray time scales—be it with decreasing certaitnty.
    Code and replication instructions are available at \href{https://github.com/syrkis/bsc}{\texttt{github.com/syrkis/bsc}}.

\end{abstract}

\section{Introduction}
The fraction of consumed media that is digital has been increasing. This presents both oppertunites and dangers. The dagners include increased probabiliy of harmful cultural positive feedback loops as the dynamicness of iblah bla la blah. At the same time, the real time analysability allows for vastly increasing the health of publiuc disoucsser.


\section{Background}
The experiment, seeking to predict distributional features of a social survey, becomes most similar in intention to precisly that.
Instead of askiing tens of thoudsands of people questionaires,
the experiment attempt to infer the state much cheaper, and with much large time granularity (though with much less specifcity).
Infering culatual states from population reading activity,
is
 the release of the Google Ngram Viewer quantitative analysis analysis of digitzed texts


The experiment at hand falls 
Extracting predictive power from text based user activity has both commerical applicability, nad is being done by arious social listening companies.

Cite. Hofstede, Jonathan Haidth, Swartz space, etc.


\section{Data}
Two experiment rests on two publicly avaiable datasets 
\emph{1)} representing cultural values; the bi-yearly Euopean Social Survey (ESS) responses from 23 countries
representing population reading activity, and
\emph{2)} representing population reading activity; the daily top 1,000 most read Wikipedia (Wiki) articles for the Wiki
project, most uniquely associated with each country
(Danish for Denmark, Hebrew for Isreal, and so on), representing cultural values

\subsection*{ESS}
The Euopean Social Survey (ESS) is a survey of the social and cultural values of European countries. 
The ESS dataset is a collection of responses from 23 countries to a series of questions about the c

\subsection*{Wiki}

The Wikipedia\footnote{\url{https://wikimedia.org/api/rest_v1/}} (Wiki) data consists of daily lists of summaries of the most viewed articles for each of the 23 Wikipedia Projects, from July 1st 2015 to December 31st 2019.
This ammounts to 1,645 days times 23 countries yileding 37,835 examples of daily Wikipedia activity.
The summaries of these articles serves as input $X$ for the model, while a vector $W$ representing how many times a given article was read represents serves as another input.

The data and an assocaited data statement is available at \href{https://data.bsc.syrkis.com}{\texttt{data.bsc.syrkis.com}}.

\subsubsection*{\emph{Dailies}}
1,311,954 unique articles across 23 languages in 1,645 days. This means that the average article is read 29 times.

\subsubsection*{\emph{Articles}}

\subsection*{European Social Survey}

The European Social Survey\footnote{\url{https://europeansocialsurvey.org/downloadwizard/}} (ESS) is a survey conducted in various European countries every second year from 2000 (round 1) to 2018 (round 9). It is conducted in person and consits of hundreds country agnostic and country specific questions. This exerpiment uses rounds 7, 8 and 9, as these overlap with the avaiable Wikipedia data.
Only questions asked in all countries were comnsidered for inclusion. A list of includefed questions can be seen in Appendix  \ref{appendix:data}


\subsection*{Efficiency}



\section{Methodology}
The experiment evaluated the folow hypothesis:
\\
\\
$H_0$ :\qquad\emph{Cultural states cannot be infered from population reading activty.}
\\
\\
$H_A$ :\qquad\emph{Cultural states can be inferred from population reading activity.}
\\
\\
A neural model was trained to predict the average and the variance of the distributions to the five most principal components of the \texttt{ESS} data for each country in the training set..

Countries present in all three ESS rounds where in the training data on which k-fold cross caliadation trauiug was used.
Countries present in less than three rounds were included in the training set. This was done so as to
ensure the applicability of the model on countries not seen before.

The experiment attempts to predict the mean and variance of five factors of ESS human value survey responses grouped by round and country from the wikipedia REeading activity within the associaed project.
The wikipedia summaries are represented by topics extracted through the use of variational auto encoders. The topic vectors along with their assocaited weights blah blah.
On notable desciosnb choice: The ESS data and the wikipedia data are on vastly different time scales. Also I want be able to place both a month or a year in Value space. Therefor I opted towards not using month matricies as trining input, but rather, similar to embedding, just look at weighted co occuruance. One sample during training is then: This articles was read this much while this was the cultural state (ESS)
We want to get at what things are uniquely heppening in this country, blah blah, thus subtracting the average country state from each country, and watching how the evolution of the factors something sometig.
Pretrained multinlingual embedding from mBERT was used to represent words \cite{artetxe-etal-2017-learning}. Langauge agnostics, semantic oriented.
\cite{kingma2013auto} i also used as topic modelling
\cite{DBLP:journals/corr/abs-1810-04805}
\cite{Wu2020}

The archtecture consists of an encoder $f$, a decoder $g$ and an inferer $h$:
$$
z  = f(x) \qquad
x' = g(z) \qquad
y' = h(z, w)
$$
Both $x'$ and $y'$ are optmised for by minimising the mean sqaured error between these and $x$ and $y$.
$f$, $g$ and $h$ are all neural models, with $f$ and $g$ together being a stadard neural variational autoencoder, which has proven to do well for topic modelling.
Using variational autoencoders for topic modelling is hardly new. What is slightly unique about the aproch is the utilising of the compressed representattion to infer our $y$.
This compressed representation aspirationally enocdes for something akin to text theme, and is weigthed by how much the articles in focus was read the day in question.
$h$, similar to $g$, conists of a two linear layers and two convelutional layers, so as to stay as convential as possible.
Notably $w \odot v$ is computed, with $w$ being the matrix representing pages views for given articles, and $v$ being a weigth optimised for.



\section{Results}
The hyper parameters found to perform the best on the validation are a batch size of 32, a sample size (how many of the first words of a Wiki article summary are included) of 16, Adam as optimizer. Training on how to weight $w$ improved accuracy by 0.5. Inspection of $v$ seems to indicate that the msot informative articles are in the long tail.

The model achives an accuracy of 0.60, which with $N=120$ yield a $p$-value of 0.01.
Splitting by variacne and average (N=60) for each yields accuracies of 0.62 ($p$ 0.04) and 0.58 ($p$ 0.004).

Spitting by factors yields the following matrix.

With $N=24$ for each factor test no siginificant p-value is reached (bernfandino correction). 

\begin{table}[H]
\centering
\begin{tabular}{|l|l|l|l|l|l|} 
\hline
factor & f1 & f2 & f3 & f4 & f5 \\ 
\hline
weight & 0.5 & 0.2 & 0.1 & 0.1 & 0.1 \\ 
\hline
average &  &  &  &  &   \\
\hline
variance &  &  &  &  &   \\
\hline
\end{tabular}
\end{table}

Though the $p$-values of the individual factors are below the significance threshold, whether the model is equally good at predicting onall five factors can be computed.
Assuming equal probability of being right within wach factor guess, are the observerd difference in accuracy unlikely.
With a $p$-value of 0.004 this is indeed found to be th case.



These resulsts can be replicated by runnning \texttt{docker run syrkis/bsc}.


\section{Discussion}
The model does have some predtive power, how ever the intepreteation seems to blah blah blah. By viewing how a given countries reading vectors differes from that of the average country, we find varaitions unique to that country. This could mean blah balh.


\section{Future work}
The assumption of a 1:1 mapping between countries and wiki projects can potentially be dropped with the imminent realease of the 10th ESS round.
The result, though statistically significant, something something ... the most immediate of which is the imminent (June 2022) release of the tenth ESS round. Rerunning the experiemnt with the inclusion of the ronfd would grow the currently very sparse ESS data by 25 percent and, further allow opportunity to disconfirm the hyptoehes is there being a correlation here.
A logical next step would be to ditch the assumption of a 1:1mappingbetween countries and wikipedia projects, in favor of the truth that is many countries us emany wikis. Data for this are avavilble from 2020 an onwatrds.
The current model design should be smoothly adaptble for a within country mulitilingual setup.
Further more moving beyond classfication towards predicting the actual means and variances of the respective factors will be ineresting.




\section{Conclusion}
The experiment, achiving and overall accuracy of 0.6 with a $p$-value of 0.003, shows that a mapping between reading activity and cutlural states \emph{can} be infered.

The models inference is within 1 stanTdard deviation of the true value component average and distribution 95\% of the time. When doing per quarter prediction the perormance does drop, though whether this is due to unquantified chagnes in cultural states or model error can not be discovered with the current experimental setup.


% Entries for the entire Anthology, followed by custom entries
\bibliography{anthology,custom}

\end{document}
