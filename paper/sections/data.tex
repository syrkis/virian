Two experiment rests on two publicly avaiable datasets 
\emph{1)} representing cultural values; the bi-yearly Euopean Social Survey (ESS) responses from 23 countries
representing population reading activity, and
\emph{2)} representing population reading activity; the daily top 1,000 most read Wikipedia (Wiki) articles for the Wiki
project, most uniquely associated with each country
(Danish for Denmark, Hebrew for Isreal, and so on), representing cultural values

\subsection*{ESS}
The Euopean Social Survey (ESS) is a survey of the social and cultural values of European countries. 
The ESS dataset is a collection of responses from 23 countries to a series of questions about the c

\subsection*{Wiki}

The Wikipedia\footnote{\url{https://wikimedia.org/api/rest_v1/}} (Wiki) data consists of daily lists of summaries of the most viewed articles for each of the 23 Wikipedia Projects, from July 1st 2015 to December 31st 2019.
This ammounts to 1,645 days times 23 countries yileding 37,835 examples of daily Wikipedia activity.
The summaries of these articles serves as input $X$ for the model, while a vector $W$ representing how many times a given article was read represents serves as another input.

The data and an assocaited data statement is available at \href{https://data.bsc.syrkis.com}{\texttt{data.bsc.syrkis.com}}.

\subsubsection*{\emph{Dailies}}
1,311,954 unique articles across 23 languages in 1,645 days. This means that the average article is read 29 times.

\subsubsection*{\emph{Articles}}

\subsection*{European Social Survey}

The European Social Survey\footnote{\url{https://europeansocialsurvey.org/downloadwizard/}} (ESS) is a survey conducted in various European countries every second year from 2000 (round 1) to 2018 (round 9). It is conducted in person and consits of hundreds country agnostic and country specific questions. This exerpiment uses rounds 7, 8 and 9, as these overlap with the avaiable Wikipedia data.
Only questions asked in all countries were comnsidered for inclusion. A list of includefed questions can be seen in Appendix  \ref{appendix:data}


\subsection*{Efficiency}

