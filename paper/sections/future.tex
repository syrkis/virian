The result, though statistically significant, something something ... the most immediate of which is the imminent (June 2022) release of the tenth ESS round. Rerunning the experiemnt with the inclusion of the ronfd would grow the currently very sparse ESS data by 25 percent and, further allow opportunity to disconfirm the hyptoehes is there being a correlation here.
A logical next step would be to ditch the assumption of a 1:1mappingbetween countries and wikipedia projects, in favor of the truth that is many countries us emany wikis. Data for this are avavilble from 2020 an onwatrds.
The current model design should be smoothly adaptble for a within country mulitilingual setup.
Further more moving beyond classfication towards predicting the actual means and variances of the respective factors will be ineresting.


