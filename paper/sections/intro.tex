Increasingly, the written word is consumed digitally.
Among the numerous implications hereof, possibility of

is the increase of the granularity with which population reading activity can be gagued.
This is equally true in academic, commerical and governmental contexts.
Various tools offer in depth analytics of user trafic patterns on websites, opening up the potential for updating content curation policies real time, based on what yields higest engagement.
This is true for

When analysed in conjunction with social surveys, population reading activity can be thought of as a symptom of what social surveys attempt to measure; cultural states.
This begets the question "\emph{can we, based on a population reading activity alone, infer something about the assocaited culture?}"

When combined with national social surveys, this increased granularity opens up the potential of infering culutural states directly.



ease and detail in the analysis of the reading habits within populations.

The fraction of consumed media that is digital has been increasing. This presents both oppertunites and dangers. The dagners include increased probabiliy of harmful cultural positive feedback loops as the dynamicness of iblah bla la blah. At the same time, the real time analysability allows for vastly increasing the health of publiuc disoucsser.
The experiment evaluated the folow hypothesis:
\\
\\
\begin{tabular}{ll}
  $H_0$ \ : & \begin{tabular}{@{}l@{}}\emph{``Cultural states can't be inferred from}\\\emph{population reading activty.''}\end{tabular} \\
            & \\
  $H_A$ \ : & \begin{tabular}{@{}l@{}}\emph{``Cultural states can be inferred from}\\\emph{population reading activty.''}\end{tabular}
\end{tabular}
