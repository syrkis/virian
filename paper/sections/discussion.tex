The findings outlines above indicate little more than there's a there there.
That there is some association between the words a culture consumes and its values is hardly a surpise.
However, what shape the mapping between these two phenomena has is less obvious.
The most tepid interpration of the results outlines above is that this mapping is shallow.
The simplicity of the model with which the p value of 0.03 was achived, as well as the quantifiably
incorrect assumption of a 1:1 mapping between language and country, are testimonies to this.
Increasing the granularity with which cultural states are measured could 
The model does have some predtive power, how ever the intepreteation seems to blah blah blah. By viewing how a given countries reading vectors differes from that of the average country, we find varaitions unique to that country. This could mean blah balh.
