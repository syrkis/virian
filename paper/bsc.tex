% This must be in the first 5 lines to tell arXiv to use pdfLaTeX, which is strongly recommended.
\pdfoutput=1
% In particular, the hyperref package requires pdfLaTeX in order to break URLs across lines.

\documentclass[11pt]{article}

% Remove the "review" option to generate the final version.
\usepackage{acl}

% Standard package includes
\usepackage{times}
\usepackage{float}
\usepackage{latexsym}
\usepackage[page]{appendix}

\usepackage[
% set width and height to a4 width and height + 6mm
width=23.6truecm, height=32.3truecm,
% use any combination of these options to add different cut markings
cross,
% set the type of TeX renderer you use
pdftex,
% center the contents
center
]{crop}


% For proper rendering and hyphenation of words containing Latin characters (including in bib files)
\usepackage[T1]{fontenc}
% For Vietnamese characters
% \usepackage[T5]{fontenc}
% See https://www.latex-project.org/help/documentation/encguide.pdf for other character sets

% This assumes your files are encoded as UTF8
\usepackage[utf8]{inputenc}

% This is not strictly necessary, and may be commented out,
% but it will improve the layout of the manuscript,
% and will typically save some space.
\usepackage{microtype}


% If the title and author information does not fit in the area allocated, uncomment the following
%
%\setlength\titlebox{<dim>}
%
% and set <dim> to something 5cm or larger.

\title{Inferring Cultural Values from Population Reading Activity\\ \texttt{\\rough working draft}\texttt{\\please don't read\\}—}

% Author information can be set in various styles:
% For several authors from the same institution:
% \author{Author 1 \and ... \and Author n \\
%         Address line \\ ... \\ Address line}
% if the names do not fit well on one line use
%         Author 1 \\ {\bf Author 2} \\ ... \\ {\bf Author n} \\
% For authors from different institutions:
% \author{Author 1 \\ Address line \\  ... \\ Address line
%         \And  ... \And
%         Author n \\ Address line \\ ... \\ Address line}
% To start a seperate ``row'' of authors use \AND, as in
% \author{Author 1 \\ Address line \\  ... \\ Address line
%         \AND
%         Author 2 \\ Address line \\ ... \\ Address line \And
%         Author 3 \\ Address line \\ ... \\ Address line}

\author{Noah Syrkis \\
  IT University of Copenhagen \\
  \texttt{noah@syrkis.com} \\}

\begin{document}
\maketitle
\begin{abstract}
    Costly international social surveys are conducted by various institutions with yearly time granularity
    \footnote{The European Social Survey and the World Value Survey release rounds every two and seven years, respectively.}.
    Here a potential method for inferring summary statistics of a given country's response to a subset of the European Social Survey (ESS), from the reading activity in that country's associated Wikipedia project, is explored.
    The ESS subset consists of 21 questions about human values, each answered by respondents on a scale from 1 to 6.
    A one-to-one mapping between countries and Wikipedia projects is assumed, while a neural model is trained to predict if a given day's reading activity comes from a country with a high or low mean and standard deviation for the ESS questions.
    The model thus attempts to answer the question ``\emph{given only a sample of a countries Wikipedia reading activity, can something be inferred about the cultural values that country?}''
    The experiment does \emph{not} yield a positive result, indicating that the assumptions on which it rests are too incorrect.
    Code and experiment replication instructions are available at
    \href{https://github.com/syrkis/bsc}{\texttt{github.com/syrkis/bsc}}.
\end{abstract}

\section{Introduction}
Increasingly, the written word is consumed digitally.
Among the numerous implications hereof, possibility of

is the increase of the granularity with which population reading activity can be gagued.
This is equally true in academic, commerical and governmental contexts.
Various tools offer in depth analytics of user trafic patterns on websites, opening up the potential for updating content curation policies real time, based on what yields higest engagement.
This is true for

When analysed in conjunction with social surveys, population reading activity can be thought of as a symptom of what social surveys attempt to measure; cultural states.
This begets the question "\emph{can we, based on a population reading activity alone, infer something about the assocaited culture?}"

When combined with national social surveys, this increased granularity opens up the potential of infering culutural states directly.



ease and detail in the analysis of the reading habits within populations.

The fraction of consumed media that is digital has been increasing. This presents both oppertunites and dangers. The dagners include increased probabiliy of harmful cultural positive feedback loops as the dynamicness of iblah bla la blah. At the same time, the real time analysability allows for vastly increasing the health of publiuc disoucsser.
The experiment evaluated the folow hypothesis:
\\
\\
\begin{tabular}{ll}
  $H_0$ \ : & \begin{tabular}{@{}l@{}}\emph{``Cultural values can't be inferred from}\\\emph{population reading activty.''}\end{tabular} \\
            & \\
  $H_A$ \ : & \begin{tabular}{@{}l@{}}\emph{``Cultural values can be inferred from}\\\emph{population reading activty.''}\end{tabular}
\end{tabular}


\section{Background}
The experiment, seeking to predict distributional features of a social survey, becomes most similar in intention to precisly that.
Instead of askiing tens of thoudsands of people questionaires,
the experiment attempt to infer the state much cheaper, and with much large time granularity (though with much less specifcity).
Infering culatual states from population reading activity,
is
 the release of the Google Ngram Viewer quantitative analysis analysis of digitzed texts


The experiment at hand falls 
Extracting predictive power from text based user activity has both commerical applicability, nad is being done by arious social listening companies.



\section{Data}
The experiment, focusing on 23 European countries (see Appendix A) between 2014 and 2020, relias on samples


The experiment infering cultural states from population reading activity utilizes data representing both. The experiment, infering cultural states from population reading activity, rests on two data sets—one representing cutlural states; the European Social Survey, the other representing population reading activity; Wikiepdia.
ygg

\subsection*{Wikipedia Projects}

\footnote{\url{https://wikimedia.org/api/rest_v1/}}
The Wikipedia (Wiki) data consists of daily lists of summaries of the most viewed articles for each of the 23 Wikipedia Projects.




Summaries of daily top 1000 read aricles for Wikipedia projects from july 1st 2015 untill 2020.

\subsection*{European Social Survey}

The European Social Survey\footnote{\url{https://europeansocialsurvey.org/downloadwizard/}} (ESS) is a survey conducted in various European countries every second year from 2000 (round 1) to 2018 (round 9). It is conducted in person and consits of hundreds country agnostic and country specific questions. This exerpiment uses rounds 7, 8 and 9, as these overlap with the avaiable Wikipedia data.
Only questions asked in all countries were comnsidered for inclusion. A list of includefed questions can be seen in Appendix C.



\section{Methodology}
A neural model was trained to predict the average and the variance of the distributions to the five most principal components of the \texttt{ESS} data for each country in the training set..

Countries present in all three ESS rounds where in the training data on which k-fold cross caliadation trauiug was used.
Countries present in less than three rounds were included in the training set. This was done so as to
ensure the applicability of the model on countries not seen before.

\subsection*{Preprocessing}
Samples are constructed on demand by from the two sources above.
The text is tokenized using BPEmb, and the tokens are further converted into multilingual embeddigns.
An iterator is then constructed outputting a tuple $(x_i, w_i, y_i)$.

\subsection*{Model architecture}
The experiment attempts to predict the mean and variance of five factors of ESS human value survey responses grouped by round and country from the wikipedia REeading activity within the associaed project.
The wikipedia summaries are represented by topics extracted through the use of variational auto encoders. The topic vectors along with their assocaited weights blah blah.
On notable desciosnb choice: The ESS data and the wikipedia data are on vastly different time scales. Also I want be able to place both a month or a year in Value space. Therefor I opted towards not using month matricies as trining input, but rather, similar to embedding, just look at weighted co occuruance. One sample during training is then: This articles was read this much while this was the cultural state (ESS)
We want to get at what things are uniquely heppening in this country, blah blah, thus subtracting the average country state from each country, and watching how the evolution of the factors something sometig.
Pretrained multinlingual embedding from mBERT was used to represent words \cite{artetxe-etal-2017-learning}. Langauge agnostics, semantic oriented.
\cite{kingma2013auto} i also used as topic modelling
\cite{DBLP:journals/corr/abs-1810-04805}
\cite{Wu2020}

The archtecture consists of an encoder $f$, a decoder $g$ and an inferer $h$:
$$
z  = f(x) \qquad
x' = g(z) \qquad
y' = h(z, w)
$$
Both $x'$ and $y'$ are optmised for by minimising the mean sqaured error between these and $x$ and $y$.
$f$, $g$ and $h$ are all neural models, with $f$ and $g$ together being a stadard neural variational autoencoder, which has proven to do well for topic modelling.
Using variational autoencoders for topic modelling is hardly new. What is slightly unique about the aproch is the utilising of the compressed representattion to infer our $y$.
This compressed representation aspirationally enocdes for something akin to text theme, and is weigthed by how much the articles in focus was read the day in question.
$h$, similar to $g$, conists of a two linear layers and two convelutional layers, so as to stay as convential as possible.
Notably $z \odot w \odot v$ is computed, with $w$ being the matrix representing pages views for given articles, and $v$ being a weigth optimised for.


\subsection*{Training}

The experiment attempts to predict the mean and variance of five factors of ESS human value survey responses grouped by round and country from the wikipedia REeading activity within the associaed project.

\subsection*{Evaluation}
Though the loss function minimises the mean squared error between the paris $(x, x')$ and $(y, y')$ the model, on inference time performers binary classification about whether a given sample of text and associated weigth comes from a country above or below the mean variance and average fro the five different factors.


\section{Results}
The model achives an accuracy of 0.60, which with $N=120$ yield a $p$-value of 0.01.
Splitting by variacne and average (N=60) for each yields accuracies of 0.62 ($p$ 0.04) and 0.58 ($p$ 0.004).

Spitting by factors yields the following matrix.

With $N=24$ for each factor test no siginificant p-value is reached (bernfandino correction). 

\begin{table}[H]
\centering
\begin{tabular}{|l|l|l|l|l|l|} 
\hline
factor & f1 & f2 & f3 & f4 & f5 \\ 
\hline
weight & 0.5 & 0.2 & 0.1 & 0.1 & 0.1 \\ 
\hline
average &  &  &  &  &   \\
\hline
variance &  &  &  &  &   \\
\hline
\end{tabular}
\end{table}

Though the $p$-values of the individual factors are below the significance threshold, whether the model is equally good at predicting onall five factors can be computed.
Assuming equal probability of being right within wach factor guess, are the observerd difference in accuracy unlikely.
With a $p$-value of 0.004 this is indeed found to be th case.

These resulsts can be replicated by runnning \texttt{docker run syrkis/bsc}.


\section{Discussion}
The model does have some predtive power, how ever the intepreteation seems to blah blah blah. By viewing how a given countries reading vectors differes from that of the average country, we find varaitions unique to that country. This could mean blah balh.


\section{Future work}
The assumption of a 1:1 mapping between countries and wiki projects can potentially be dropped with the imminent realease of the 10th ESS round.
The result, though statistically significant, something something ... the most immediate of which is the imminent (June 2022) release of the tenth ESS round. Rerunning the experiemnt with the inclusion of the ronfd would grow the currently very sparse ESS data by 25 percent and, further allow opportunity to disconfirm the hyptoehes is there being a correlation here.
A logical next step would be to ditch the assumption of a 1:1mappingbetween countries and wikipedia projects, in favor of the truth that is many countries us emany wikis. Data for this are avavilble from 2020 an onwatrds.
The current model design should be smoothly adaptble for a within country mulitilingual setup.
Further more moving beyond classfication towards predicting the actual means and variances of the respective factors will be ineresting.




\section{Conclusion}
T
The models inference is within 1 stanTdard deviation of the true value component average and distribution 95\% of the time. When doing per quarter prediction the perormance does drop, though whether this is due to unquantified chagnes in cultural states or model error can not be discovered with the current experimental setup.


% Entries for the entire Anthology, followed by custom entries
\bibliography{anthology,custom}

\onecolumn
\begin{appendices}

The experiment, focusing on 23 European countries (see Appendix A) between 2014 and 2020, relias on samples


The experiment infering cultural states from population reading activity utilizes data representing both. The experiment, infering cultural states from population reading activity, rests on two data sets—one representing cutlural states; the European Social Survey, the other representing population reading activity; Wikiepdia.
ygg

\subsection*{Wikipedia Projects}

\footnote{\url{https://wikimedia.org/api/rest_v1/}}
The Wikipedia (Wiki) data consists of daily lists of summaries of the most viewed articles for each of the 23 Wikipedia Projects.




Summaries of daily top 1000 read aricles for Wikipedia projects from july 1st 2015 untill 2020.

\subsection*{European Social Survey}

The European Social Survey\footnote{\url{https://europeansocialsurvey.org/downloadwizard/}} (ESS) is a survey conducted in various European countries every second year from 2000 (round 1) to 2018 (round 9). It is conducted in person and consits of hundreds country agnostic and country specific questions. This exerpiment uses rounds 7, 8 and 9, as these overlap with the avaiable Wikipedia data.
Only questions asked in all countries were comnsidered for inclusion. A list of includefed questions can be seen in Appendix C.


\label{appendix:data}
\clearpage

\end{appendices}



\end{document}
